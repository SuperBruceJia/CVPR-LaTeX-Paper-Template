\section{Introduction}
\label{sec:intro}
The introduction can be considered as the expansion of the abstract. Usually, the introduction consists of the following four parts:
\begin{enumerate}[itemsep=0pt,parsep=0pt,topsep=2bp]
    \item Introduce what is the problem and its significance.
    \item Introduce the most related existing methods and point out their limitations.
    \item Introduce the motivation and key idea of the proposed method, and illustrate the novelty and advantages of our method over existing methods.
    \item Summarize the key contributions of the paper from the aspects of \emph{problem formulation}, \emph{proposed method}, \emph{created dataset}, \emph{technical novelty}, \emph{experimental results}, etc. Normally, the contribution list should not exceed three points.
\end{enumerate}

\begin{figure}[tb] \centering
    \includegraphics[width=0.48\textwidth,height=0.3\textwidth]{example-image}
    \caption{Teaser image. Usually, we will have a teaser image on the first page. It can be the key idea of the proposed method or some eye-catching results.} \label{fig:figure1}
\end{figure}

\vspace{3pt}
If you do not know how to start at the very beginning, you may find a most related paper and follow its structure. But please remember \textbf{DO NOT COPY ANY SENTENCES} from other papers, otherwise, you and the coauthors will be in big trouble!

\vspace{3pt}
\paragraph{Mistakes to avoid} When writing your manuscript, please avoid the following common mistakes:

\begin{enumerate}[itemsep=0pt,topsep=2bp]
    \item The first character in a sentence should be capitalized:\\
        \wrong{how are you?}\\
        \correct{How are you?}
    \item There should be a space before the open parentheses:\\
        \wrong{Convolutional neural network(CNN) has been successfully applied on various vision problems.}\\
        \correct{Convolutional neural network (CNN) has been successfully applied on various vision problems.}
    \item There should be a space before the citation:\\
        \wrong{A proposes a method B for this problem[1].}\\
        \correct{A proposes a method B for this problem~[1].}
    \item Double quotation marks should be correctly typed:\\
        \wrong{Are you "okay"?}\\
        \correct{Are you ``okay''?}
    \item There should be no space before the period and comma punctuation marks:\\
        \wrong{Convolutional neural network (CNN) has been successfully applied on various vision problems .}\\
        \correct{Convolutional neural network (CNN) has been successfully applied on various vision problems.}
    \item All equations should be numbered and there should be a punctuation (. or ,) at the end of the equation:
        \begin{align}
            \label{eq:eq1}
            & \textcolor{MyDarkRed}{E = mc^2 \quad \times} \\
            \nonumber
            & \textcolor{MyDarkRed}{E = mc^2. \quad \times}\\ 
            & \textcolor{MyDarkBlue}{E = mc^2. \quad \checkmark}
        \end{align}
\end{enumerate}


\paragraph{Rules to follow} The following are some rules to follow:
\begin{enumerate}[itemsep=0pt,topsep=2bp]
    \item Define a macro for a word or phrase if it appears frequently (\eg, the method name and the dataset name). The command can be\\``\textbackslash newcommand\{\textbackslash NetName\}\{A Great Deep Net\}''.
    \item Use ``\textbackslash ie'' command for ``\ie'' and use ``\textbackslash eg'' for ``\eg.''.
    \item Refer a table with ``\Tref{tab:table1}'' instead of Tab.~1.
    \item When referring to a figure, use ``Figure~\ref{fig:figure1}'' at the beginning of a sentence and ``Fig.~\ref{fig:figure1}'' elsewhere.\\
        \wrong{Fig.1 shows our results.}\\
        \correct{Our results are shown in Fig.~1.}\\
        \correct{Figure~1 shows our results.}
    \item The table caption should be at the top.
    \item The figure caption should be at the bottom.
\end{enumerate}

%\textcolor{\dummycolor}{\lipsum[1]}

\vspace{3pt}
\paragraph{Example of Contribution Summary} In summary, the key contributions of this paper are as follows:  
\begin{itemize}[itemsep=0pt,parsep=0pt,topsep=2bp]
    \item We propose a two-stage framework, which first performs image alignment and HDR fusion in the image space and then in feature space, for HDR video reconstruction from sequences with alternating exposures.
    \item We create a real-world video dataset captured with alternating exposures as a benchmark to enable quantitative evaluation for this problem. 
    \item Our method achieves state-of-the-art results on both synthetic and real-world datasets.
\end{itemize}


